%-------------------------
% Resume in Latex
% Author : Jake Gutierrez
% Based off of: https://github.com/jakeryang/resume
% License : MIT
%------------------------

\documentclass[letterpaper,11pt]{article}

\usepackage{latexsym}
\usepackage[empty]{fullpage}
\usepackage{titlesec}
\usepackage{marvosym}
\usepackage[usenames,dvipsnames]{color}
\usepackage{verbatim}
\usepackage{graphicx}
\usepackage{enumitem}
\usepackage[hidelinks]{hyperref}
\usepackage{fancyhdr}
\usepackage[english]{babel}
\usepackage{tabularx}
\usepackage{multicol}
\usepackage{longtable}
\input{glyphtounicode}


%----------FONT OPTIONS----------
% sans-serif
% \usepackage[sfdefault]{FiraSans}
% \usepackage[sfdefault]{roboto}
% \usepackage[sfdefault]{noto-sans}
% \usepackage[default]{sourcesanspro}

% serif
% \usepackage{CormorantGaramond}
% \usepackage{charter}


\pagestyle{fancy}
\fancyhf{} % clear all header and footer fields
\fancyfoot{}
\renewcommand{\headrulewidth}{0pt}
\renewcommand{\footrulewidth}{0pt}

% Adjust margins
\addtolength{\oddsidemargin}{-0.5in}
\addtolength{\evensidemargin}{-0.5in}
\addtolength{\textwidth}{1in}
\addtolength{\topmargin}{-0.3in}
\addtolength{\textheight}{1.0in}

\urlstyle{same}

\raggedbottom
\raggedright
\setlength{\tabcolsep}{0in}

% Sections formatting
\titleformat{\section}{
  \vspace{-4pt}\scshape\raggedright\large
}{}{0em}{}[\color{black}\titlerule \vspace{-5pt}]

% Ensure that generate pdf is machine readable/ATS parsable
\pdfgentounicode=1

%-------------------------
% Custom commands
\newcommand{\resumeItem}[1]{
  \item\small{
    {#1 \vspace{-2pt}}
  }
}

\newcommand{\resumeItemNull}[1]{
  \item[]\small{
    {#1 \vspace{-2pt}}
  }
}

\newcommand{\resumeItemInfo}[2]{
  \item[#1]\small{
    {#2 \vspace{-2pt}}
  }
}

\newcommand{\resumeSubheading}[4]{
  \vspace{-2pt}\item
    \begin{tabular*}{0.97\textwidth}[t]{l@{\extracolsep{\fill}}r}
      \textbf{#1} & #2 \\
      \textit{\small#3} & \textit{\small #4} \\
    \end{tabular*}\vspace{-7pt}
}

\newcommand{\resumeSubheadingSmall}[2]{
  \vspace{-2pt}\item
    \begin{tabular*}{0.97\textwidth}[t]{l@{\extracolsep{\fill}}r}
      \small #1 & \small #2 \\
    \end{tabular*}\vspace{-12pt}
}



\newcommand{\resumeSubheadingInfo}[5]{
  \vspace{-2pt}\item
    \begin{tabular*}{0.97\textwidth}[t]{l@{\extracolsep{\fill}}r}
      \textbf{#1} & #2 \\
      \textit{\small#3} & \textit{\small #4} \\
      \textsc{~#5} & \\
    \end{tabular*}\vspace{-7pt}
}

\newcommand{\resumeSubSubheading}[2]{
    \item
    \begin{tabular*}{0.97\textwidth}{l@{\extracolsep{\fill}}r}
      \textit{\small#1} & \textit{\small #2} \\
    \end{tabular*}\vspace{-7pt}
}

\newcommand{\resumeProjectHeading}[2]{
    \item
    \begin{tabular*}{0.97\textwidth}{l@{\extracolsep{\fill}}r}
      \small#1 & #2 \\
    \end{tabular*}\vspace{-7pt}
}
    

\newcommand{\resumeSubItem}[1]{\resumeItem{#1}\vspace{-4pt}}

\renewcommand\labelitemii{$\vcenter{\hbox{\tiny$\bullet$}}$}

\newcommand{\resumeSubHeadingListStart}{\begin{itemize}[leftmargin=0.15in, label={}]}
\newcommand{\resumeSubHeadingListEnd}{\end{itemize}}
\newcommand{\resumeItemListStart}{\begin{itemize}}
\newcommand{\resumeItemListEnd}{\end{itemize}\vspace{-4.5pt}}

%-------------------------------------------
%%%%%%  RESUME STARTS HERE  %%%%%%%%%%%%%%%%%%%%%%%%%%%%


\begin{document}

%----------HEADING----------
% \begin{tabular*}{\textwidth}{l@{\extracolsep{\fill}}r}
%   \textbf{\href{http://sourabhbajaj.com/}{\Large Sourabh Bajaj}} & Email : \href{mailto:sourabh@sourabhbajaj.com}{sourabh@sourabhbajaj.com}\\
%   \href{http://sourabhbajaj.com/}{http://www.sourabhbajaj.com} & Mobile : +1-123-456-7890 \\
% \end{tabular*}

\begin{center}
    \textbf{\Huge \scshape Rashad Eletreby} \\ \vspace{1pt}
    \small 650-714-2627 $|$ \href{mailto:eletreby.rashad@gmail.com}{\underline{eletreby.rashad@gmail.com}} $|$ 
    \href{https://www.linkedin.com/in/reletreby}{\underline{https://www.linkedin.com/in/reletreby}} 
    \\ \vspace{1mm}
    \small H-1B Status $|$ Pending I-485
\end{center}


%-----------EDUCATION-----------
\section{Education}
  \resumeSubHeadingListStart
    \resumeSubheading
      {Carnegie Mellon University}{Pittsburgh, PA}
      {PhD in Electrical and Computer Engineering}{Aug. 2015 -- Sep. 2019}
      
     \vspace{1mm}
    \resumeSubheading
      {Carnegie Mellon University}{Pittsburgh, PA}
      {MS in Electrical and Computer Engineering}{Aug. 2015 -- Sep. 2019}
      
       \vspace{1mm}
    \resumeSubheading
      {Cairo University}{Cairo, Egypt}
      {MS in Electrical and Computer Engineering}{Sep. 2012 -- July 2014}
      
      \vspace{1mm}
    \resumeSubheading
      {Cairo University}{Cairo, Egypt}
      {BS in Electrical and Computer Engineering}{Sep. 2007 -- July 2012}
  \resumeSubHeadingListEnd


%\vspace{0.0001mm}
%-----------PROGRAMMING SKILLS-----------
\vspace{1mm}
\section{Technical Skills}
 \begin{itemize}[leftmargin=0.15in, label={}]
    \small{\item{
     \textbf{Coding}{: Python, C++, Hive, SQL} \\
     \textbf{Developer Tools}{: Git, Jupyter, Google Cloud Platform, VS Code, PyCharm, IntelliJ} \\
     \textbf{Libraries}{: xgboost, PySpark, pandas, Keras, tensorflow, NumPy, Matplotlib, scikit-learn, igraph, networkx, joblib, requests, SciPy}
    }}
 \end{itemize}
 
%-----------EXPERIENCE-----------
\section{Experience}
  \resumeSubHeadingListStart

    \resumeSubheading
      {Staff Data Scientist}{December 2020 -- Present}
      {Walmart Inc.}{Hoboken, NJ}
      \resumeItemListStart
      \resumeItem{{\bf Search Listing Diversity:} Designing algorithms based on image understanding to increase the diversity of search results and decrease duplicate listings at Walmart.com.}

        \resumeItem{{\bf Learning to Rank:} 
         \begin{itemize}
         \item Full stack design of advanced machine learning models in the context of learning to rank to improve the relevance and conversion of search and browse results.
        \item Developed the first Learning to Rank (LETOR) framework for baseline ranking layer that serves all of Walmart.com browse shelves. The developed framework led to 0.37\% GMV lift. The project drove an additional \$XxM in revenue.
        \item Designed the first multi-objective LETOR framework to empower a unified search experience in support of Walmart OneApp launch. The Designed model achieved relevance improvements of 15\%.
        \item Created a LETOR framework that serves all of Walmart.com browse shelves. The rerank model induced by this framework led to 0.9\% GMV lift and 2\% discoverability lift. The project drove an additional \$XxM in revenue.
   \end{itemize}
    }
        \resumeItem{{\bf Experimentation:}
        \begin{itemize}
        \item Created a powerful offline tool that measures the performance of a feature in terms of GMV coverage. The tool played a crucial role in many business decisions across multiple teams whenever an online A/B test was not possible
        \item Running and analyzing customer facing experiments through interleaving and A/B testing.
	\end{itemize}
	}
      \resumeItemListEnd
      
      \vspace{1mm}
       \resumeSubheading
      {Senior Data Scientist}{October 2019 -- December 2020}
      {Walmart Inc.}{Hoboken, NJ}
      \resumeItemListStart
        \resumeItem{{\bf Learning to Rank:} 
       \begin{itemize}
        \item Developed a LETOR framework that serves all of Walmart.com browse shelves. The new model improved relevance by 5\% and increased Ads revenue by 6.3\%.
        \item Improved key features used within the LETOR framework that servers all search queries at Walmart.com. The ranking model trained on the improved features yielded 12\% improvements with respect to item discoverability.
           \item Implemented fundamental algorithmic changes to the LETOR framework at Walmart to alleviate presentation bias and accurately model customer engagement. The ranking model induced by these changes led to 0.60\% GMV lift and 5\% latency improvement. The project drove an additional \$XxM in revenue.
        \end{itemize}
    }
        \resumeItem{{\bf Experimentation:}}
        \begin{itemize}
        \item Implemented key enhancements to an offline tool that simulates online interleaving tests. The enhancements provided several key insights that aided in the decision making of many features.
        \item Developed an offline tool that examines A/B and interleaving test data and provides insights on the key segments for which the treatment performed poorly as compared to the control. The tool has been utilized by several members in the organization and is considered as a starting point for Root Cause Analysis (RCA).
        	\end{itemize}
      \resumeItemListEnd
      


    \resumeSubheading
      {Research Assistant}{Aug. 2015 -- Sep. 2019}
      {Carnegie Mellon University}{Pittsburgh, PA}
      \resumeItemListStart
      \resumeItem{{\bf Discovering Social Circles}: 
Proposed methods for automatic community detection on social network subgraphs under the {\em social circle analysis} category. The proposed methods combine structural information (graph connectivity) and content information (traits pertaining to each node in the network) to determine communities within social network graphs}
        \resumeItem{{\bf Random Graph Theory}: Proposed novel random graph models that capture the secure connectivity of large-scale, heterogeneous wireless sensor networks. Analyzed the absence of isolated nodes, connectivity, minimum node degree, and $k$-connectivity of the proposed graphs by means of rigorous mathematical proofs and computer simulations}
        \resumeItem{{\bf Network Science}: Proposed and analyzed mathematical and simulation models that characterize the role of evolutionary adaptations in facilitating the spread of information and infectious diseases in real-world complex networks}
        \resumeItem{{\bf Internet of Things}: Worked on the design, evaluation, and implementation of novel techniques that aim to i) disentangle and decode large numbers of interfering LP-WAN transmissions at a simple, single-antenna LP-WAN base station, and ii) extend the transmission range of the current LP-WAN sensors}
    \resumeItemListEnd
    

\resumeSubheading
      {Research Assistant}{Aug. 2014 -- May 2015}
      {University of Arizona}{Tucson, AZ}
      \resumeItemListStart
      \resumeItem{{\bf Physical-layer security}: Conducted research on physical layer security in multi-link wireless networks using artificial noise techniques}
    \resumeItemListEnd

  \resumeSubHeadingListEnd


%-----------GRADUATE COURSES-----------
\section{Graduate Courses}
    \resumeSubHeadingListStart
      \resumeProjectHeading
          {\textbf{Carnegie Mellon University} (GPA: 4.0)}{}
          \resumeItemListStart
            \resumeItemNull{Introduction to Machine Learning (PhD), Applied Stochastic Processes, Estimation and Detection, Wireless Communications, Game Theory}
          \resumeItemListEnd
                    
          \resumeProjectHeading
          {\textbf{University of Arizona} (GPA: 4.0)}{}
          \resumeItemListStart
            \resumeItemNull{Computer System and Network Evaluation, Advanced Topics in Computer Networks, Theory of Graphs and Networks}
          \resumeItemListEnd
                    
               \resumeProjectHeading
          {\textbf{Cairo University} (GPA: 4.0)}{}
          \resumeItemListStart
            \resumeItemNull{Optimization Methods, Advanced Mathematics, Linear/Non-Linear Control Systems}
          \resumeItemListEnd
          
          
    \resumeSubHeadingListEnd
    
   
%-----------CERTIFICATIONS-----------
\section{Certifications}
\resumeItemListStart
 \resumeItem{Neural Networks and Deep Learning - Coursera}
  \resumeItem{Improving Deep Neural Networks: Hyper-parameter Tuning, Regularization, and Optimization - Coursera}
  \resumeItemListEnd


\section{Honors and Awards}
  \resumeSubHeadingListStart
  \resumeSubheadingSmall{$\bullet$ Philip and Marsha Dowd Fellowship, Carnegie Mellon University}{August 2017 - May 2018}
  \resumeSubheadingSmall{$\bullet$ CMU Presidential Fellowship, Carnegie Mellon University}{August 2017 - May 2018}
  \resumeSubheadingSmall{$\bullet$ William J. Happel Fellowship, Carnegie Mellon University}{August 2015 - July 2016}
  \resumeSubheadingSmall{$\bullet$ Carnegie Institute of Technology Fellowship, Carnegie Mellon University}{August 2015 - July 2016}
    \resumeSubHeadingListEnd
\section{Patents}
\resumeItemListStart


\resumeItemInfo{(P3)}{\textbf{R. Eletreby}, C. Mu, Z. Wang and R. Mukherjee \textit{``Systems and Methods for Improving eCommerce Search Ranking Via Labelling Enhancements in LETOR”} - patent pending}
\resumeItemInfo{(P2)}{\textbf{R. Eletreby}, D. Zhang, S. Kumar and O. Ya\u{g}an \textit{``Empowering Low-Power Wide Area Networks in Urban Settings”} - patent pending}
\resumeItemInfo{(P1)}{M. Krunz, B. Akgun, P. Siyari, H. Rahbari, \textbf{R. Eletreby}, and O. Koyluoglu \textit{``Systems and methods for securing wireless communications"} - patent granted by USPTO}
\resumeItemListEnd


\section{Publications}

\subsubsection*{\quad \large Technical Reports}
\resumeItemListStart
\resumeItemInfo{(T1)}{\textbf{R. Eletreby} and  M. Blanco \textit{``Social Circle Analysis via Content and Structure Augmentation"} - Source code and report are available at: \url{https://github.com/reletreby/structureAug}}
\resumeItemListEnd

\subsubsection*{\quad \large Journal Papers}

\resumeItemListStart
\resumeItemInfo{(J6)}{\textbf{R. Eletreby} and O. Ya\u{g}an \textit{``Secure Connectivity of Heterogeneous Wireless Sensor Networks Under a Heterogeneous On-Off Channel Model"} - IEEE Transactions on Control of Network Systems (submitted)}

\resumeItemInfo{(J5)}{\textbf{R. Eletreby}, Y. Zhuang, K. M. Carley, O. Ya\u{g}an, and H.  Vincent Poor \textit{``The Effects of Evolutionary Adaptations on Spreading Processes in Complex Networks"} - Proceedings of the National Academy of Sciences (March 2020)}


\resumeItemInfo{(J4)}{\textbf{R. Eletreby} and O. Ya\u{g}an \textit{``Connectivity of Inhomogeneous Random K-out Graphs"} - IEEE Transactions on Information Theory (June 2020)}


\resumeItemInfo{(J3)}{\textbf{R.Eletreby} and O.Ya\u{g}an \textit{``$k$-connectivity of Inhomogeneous Random Key Graphs with Unreliable Links"} - IEEE Transactions on Information Theory (January 2019)}

\resumeItemInfo{(J2)}{\textbf{R.Eletreby} and O.Ya\u{g}an \textit{``Connectivity of Wireless Sensor Networks Secured by Heterogeneous Key Predistribution Under an On/Off Channel Model"} - IEEE Transactions on Control of Network Systems (February 2018)}


\resumeItemInfo{(J1)}{\textbf{R.Eletreby}, H.Elsayed and M.Khairy \textit{``Optimal Spectrum Assignment for Cognitive Radio Sensor Networks Under Coverage Constraint"} - IET Communications Journal (December 2014)}

\resumeItemListEnd



\subsubsection*{\quad \large Conference Papers}
\resumeItemListStart
\resumeItemInfo{(C15)}{A.Sridhar, O.Ya\u{g}an, \textbf{R.Eletreby}, S.Levin, J.B.Plotkin, and H.V. Poor \textit{``Leveraging A Multiple-Strain Model with Mutations in Analyzing the Spread of Covid-19"} - IEEE ICASSP 2021}


\resumeItemInfo{(C14)}{O.Ya\u{g}an, A.Sridhar, \textbf{R.Eletreby}, S.Levin, J.B.Plotkin, and H.V.Poor \textit{``Modeling and Analysis of the Spread of COVID-19 under a Multiple-strain Model with Mutations"} - HDSR 2021}

\resumeItemInfo{(C13)}{M.Sood, A.Sridhar, \textbf{R.Eletreby}, C.W.Wu, H.V.Poor, and O.Ya\u{g}an \textit{``Epidemic Spreading of Mutating Contagions over Multi-Layer Contact Networks"} - NetSci 2021}

\resumeItemInfo{(C12)}{\textbf{R.Eletreby} and O.Ya\u{g}an \textit{``On the Connectivity of Inhomogeneous Random $K$-out Graphs"} - IEEE ISIT 2019}

\resumeItemInfo{(C11)}{\textbf{R.Eletreby} and O.Ya\u{g}an \textit{``Connectivity of Wireless Sensor Networks Secured by the Heterogeneous Random Pairwise Key Predistribution Scheme"} - IEEE CDC 2018}


\resumeItemInfo{(C10)}{\textbf{R.Eletreby}, Y.Zhuang and O.Ya\u{g}an \textit{``Evolution of Spreading Processes on Complex Networks"} - Conference on Complex Systems 2018}


\resumeItemInfo{(C9)}{\textbf{R.Eletreby}, Y.Zhuang and O.Ya\u{g}an \textit{``Evolution of Spreading Processes on Complex Networks"} - IEEE ITA 2018 - Invited Abstract}


\resumeItemInfo{(C8)}{\textbf{R.Eletreby}, D.Zhang, S.Kumar and O.Ya\u{g}an \textit{``Empowering Low-Power Wide Area Networks in Urban Settings"} - ACM SIGCOMM 2017}


\resumeItemInfo{(C7)}{\textbf{R.Eletreby} and O.Ya\u{g}an \textit{``Secure and Reliable Connectivity in Heterogeneous Wireless Sensor Networks"} - IEEE ISIT 2017}


\resumeItemInfo{(C6)}{\textbf{R.Eletreby} and O.Ya\u{g}an \textit{``Connectivity of Inhomogeneous Random Key Graphs Intersecting Inhomogeneous Erd\H{o}s-R\'enyi Graphs"} - IEEE ISIT 2017}


\resumeItemInfo{(C5)}{\textbf{R.Eletreby} and O.Ya\u{g}an \textit{``On the network reliability problem of the heterogeneous key predistribution scheme"} - IEEE CDC 2016}


\resumeItemInfo{(C4)}{\textbf{R.Eletreby} and O.Ya\u{g}an \textit{``Node isolation of secure wireless sensor networks under a heterogeneous channel model"} - Allerton 2016}


\resumeItemInfo{(C3)}{\textbf{R.Eletreby} and O.Ya\u{g}an \textit{`` Minimum Node Degree in Inhomogeneous Random Key Graphs With Unreliable Links"} - IEEE ISIT 2016}


\resumeItemInfo{(C2)}{ \textbf{R.Eletreby}, H.Rahbari, M.Krunz \textit{``Supporting PHY-layer Security in Multi-link Wireless Networks Using Friendly Jamming"} - IEEE GLOBECOM 2015}


\resumeItemInfo{(C1)}{\textbf{R.Eletreby}, H.Elsayed and M.Khairy \textit{``CogLEACH: A Spectrum-Aware Clustering Protocol for Cognitive Radio Sensor Networks"} - CROWNCOM 2014}

\resumeItemListEnd



%-------------------------------------------
\end{document}